
\section{Introducción}

Las alas volantes son alternativas a los aviones de configuración estándar que ofrecen menor resistencia aerodinámica. No obstante, presentan ciertas complicaciones para lograr estabilidad longitudinal. En este proyecto se estudia y simula el ala volante Horten IV y su estabilidad en vuelo. Para las simulaciones de la aerodinámica de la aeronave se sigue el método de Weissinger utilizando vórtices herradura (\emph{Horseshoe Vortices}).


